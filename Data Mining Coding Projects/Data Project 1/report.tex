\documentclass[11pt,letterpaper]{article}

% Packages
\usepackage[margin=1in]{geometry}
\usepackage{booktabs}

\title{\textbf{Apriori Algorithm for Frequent Itemset Mining}\\
       \large CS470  Data Mining}
\author{Gerardo Gomez}
\date{\today}

\begin{document}

\maketitle

\section{Algorithmic Optimizations}

Several optimization techniques were implemented to achieve efficient performance:

\subsection{Transaction Pruning}
After finding frequent 1-itemsets, all infrequent items were removed from transactions. This reduces transaction sizes and speeds up subsequent operations. At each iteration, transactions that were too small to contain k-itemsets were also removed.

\subsection{Bitmap Matrix Representation}
The transaction database was represented as a binary matrix where each row is a transaction and each column is an item. This enables fast membership testing and vectorized operations using NumPy.

\subsection{Vectorized Support Counting}
Instead of using slow Python loops, NumPy's bitwise AND operations on matrix columns were used to count support for candidates. This is significantly faster than naive implementations.

\subsection{Efficient Candidate Generation}
The F(k-1) × F(k-1) method with lexicographic ordering was used to generate candidates efficiently and avoid duplicates.

\subsection{Apriori Property Pruning}
Before counting support, all subsets of each candidate were verified to be frequent. This eliminates many candidates and reduces expensive computations.

\section{Experimental Results}

\subsection{T10I4D100K Dataset}
The implementation was tested on the T10I4D100K dataset with varying minimum support values:

\begin{table}[h]
\centering
\caption{Results on T10I4D100K dataset}
\begin{tabular}{@{}cccc@{}}
\toprule
\textbf{Min Support} & \textbf{Frequent Itemsets} & \textbf{Max Size} & \textbf{Time (minutes)} \\ \midrule
3000 & 60   & 1 & 0.04  \\
2000 & 155  & 1 & 0.64  \\
1000 & 385  & 3 & 4.24  \\
500  & 1073 & 5 & 7.83  \\ \bottomrule
\end{tabular}
\end{table}

The implementation successfully completed the required test case (support=500) in 7.83 minutes, well under the 15-minute limit.

\subsection{Chess Dataset}
The Chess dataset (3,196 transactions) was tested with minimum support of 2900. The results:
\begin{itemize}
    \item Frequent itemsets found: 473
    \item Maximum itemset size: 7 items
    \item Execution time: 0.08 seconds
\end{itemize}

\subsection{Mushroom Dataset}
The Mushroom dataset (8,124 transactions) was tested with minimum support of 2000. The results:
\begin{itemize}
    \item Frequent itemsets found: 2,367
    \item Maximum itemset size: 11 items  
    \item Execution time: 0.51 seconds
\end{itemize}

\section{Experiences and Lessons Learned}

My initial implementation without optimizations was extremely slow, which taught me that for large datasets, algorithmic optimizations are essential. Using NumPy's vectorized operations instead of Python loops improved performance by approximately 6x, demonstrating how critical performance considerations are when working with large-scale data mining tasks.

Implementing Apriori from scratch gave me a much deeper understanding than just reading about it. I now truly understand why the Apriori property is so powerful for pruning the search space and how candidate generation works. This hands-on experience was invaluable in cementing my understanding of the algorithm's mechanics.

I learned the importance of testing with various support values during development. Starting with high support values allowed me to quickly verify correctness before optimizing for lower support values that take longer to run. Choosing the right data structures also made a huge difference. Using frozensets for itemsets, dictionaries for fast lookups, and NumPy arrays for the bitmap matrix were all crucial for achieving good performance.

I developed the solution iteratvely, first getting a correct but slow version working,then optimizing the slowest parts. This approach was much more effective than trying to optimize everything at once. The iterative development process helped me understand which optimizations provided the most benefit and allowed me to make informed decisions about where to focus my efforts.

\end{document}

